\documentclass{article}
\usepackage{amsmath, amssymb}
\begin{document}
Assignment 5\\
Ananya Jana, netID: aj611\\
\\
{\bf Q.2}\\
The modified Newton's program for 2 variables is {\it modifiednewtontwovar.m}.\\
{\bf a)}\\
The solution is: Value of x and y: \\
   x = ${\pm}$1.581138830084190\\
   y = ${\pm}$1.224744871391589\\
\\
Below, we have shown the results for the initial point(x, y) = (1, 1).\\
\\
Attaching the matlab output snippet for the modifiednewtontwovar.m\\
Choose 1 for part a, Choose 2 for part b: 1\\
Number of iterations: 15\\
   1.750000000000000\\
   1.250000000000000\\
\\
   0.790569415042095\\
\\
   1.589285714285714
   1.225000000000000\\
\\
   0.162647107667653\\
\\
   1.581159711075441\\
   1.224744897959184\\
\\
   0.008130006471375\\
\\
   1.581138830222068\\
   1.224744871391589\\
\\
     2.088087027479742e-05\\
\\
   1.581138830084190\\
   1.224744871391589\\
\\
Iteration number:\\ 
     3\\
\\
Value of x and y: \\
   1.581138830084190\\
   1.224744871391589\\
\\
When we take other starting values of (x, y) as (1, -1), (-1, 1) and (-1, -1), we can obtain the other solutions using the same program.\\
\\
{\bf b)}\\
The solution is: Value of x and y: \\
x = 1.945026819131982\\
   y = 0.673007169617913\\
\\
Attaching the matlab output snippet for the modifiednewtontwovar.m\\
Choose 1 for part a, Choose 2 for part b: 2\\
Number of iterations: 15\\
   2.250000000000000\\
   0.375000000000000\\
\\
   0.976281209488332\\
\\
   1.993518518518518\\
   0.624537037037037\\
\\
   0.357842819120573\\
\\
   1.946743715555195\\
   0.671290273402707\\
\\
   0.066134312598847\\
\\
   1.945029130257479\\
   0.673004858492416\\
\\
   0.002424789634791\\
\\
   1.945026819136181\\
   0.673007169613714\\
\\
     3.268419084416970e-06\\
\\
   1.945026819131982\\
   0.673007169617913\\

Iteration number: \\
     4\\
\\
Value of x and y: \\
   1.945026819131982\\
   0.673007169617913\\
\\
{\bf c)} The generalized program for 3 variables is {\it generalnewtonthreevar.m}\\
We use 4 initial starting points and obtain the corresponding roots of the given equation for those.\\
Initial point: (1, 1, 1)\\
Roots: x = 2.449489742783178, y = 2.449489742783178, z = 2\\
\\
Initial point: (1, -2, -1)\\
Roots: x = 0.585786437626905, y = -3.414213562373095, z = -2\\
\\
Initial point: (-1, -1, 1)\\
Roots: x = -2.449489742783178, y = -2.449489742783178, z = 2\\
\\
Initial point: (2, -1, 11)\\
Roots: x = 3.414213562373095, y = -0.585786437626905, z = -2\\
\\
\\
{\bf Q.1}\\
{\bf a)} We have, $z^{(0)} = \sum_{j=1}^{n}{a_j}{x_j}$  where $x_j$ are the Eigen vectors corresponding to Eigen values ${\lambda}_j$.\\ \\
${A^m}{z^{(0)}} = {A^m}\sum_{j=1}^{n}{a_j}{x_j} = \sum_{j=1}^{n}{a_j}{A^m}{x_j}$\\
{\bf Proof:}\\
We will prove that ${A^m}{z^{(0)}} = \sum_{j=1}^{n}{a_j}{{{\lambda}_j}^m}{x_j}$ by using Induction and then move on to the next part of the proof.\\
Basis$(m = 1)$:\\
${A^m}{z^{(0)}} = {A}\sum_{j=1}^{n}{a_j}{x_j} $\\
$= \sum_{j=1}^{n}{a_j}{A}{x_j}$\\ 
$= \sum_{j=1}^{n}{a_j}{{\lambda}_j}{x_j}$\\
This is because of the fact that for the j{\it th} Eigen value and Eigen vector, we have ${A}{x_j} = {{\lambda}_j}{x_j}$\\
\\
Let us assume that ${A^m}{z^{(0)}} = \sum_{j=1}^{n}{a_j}{{{\lambda}_j}^m}{x_j}$  holds good for $m = k$.
Hence, ${A^k}{z^{(0)}} = \sum_{j=1}^{n}{a_j}{{{\lambda}_j}^k}{x_j}$   -Eqn.(1)\\
\\
Let us check this for $m = k + 1$.\\
${A^{k+1}}{z^{(0)}} = A.{A^k}{z^{(0)}}$\\
$ = A\sum_{j=1}^{n}{a_j}{{{\lambda}_j}^k}{x_j}$ [From Eqn. (1)]\\
$ = \sum_{j=1}^{n}{a_j}{{{\lambda}_j}^k}A{x_j}$\\
$ = \sum_{j=1}^{n}{a_j}{{{\lambda}_j}^k}{{\lambda}_j}{x_j}$ [As we already know ${A}{x_j} = {{\lambda}_j}{x_j}$]\\
$ = \sum_{j=1}^{n}{a_j}{{{\lambda}_j}^{k+1}}{x_j}$\\
\\
Hence we proved by Induction that ${A^m}{z^{(0)}} = \sum_{j=1}^{n}{a_j}{{{\lambda}_j}^m}{x_j}$ is true.
\\
We have ${A^m}{z^{(0)}} = \sum_{j=1}^{n}{a_j}{{{\lambda}_j}^m}{x_j}$\\
$ = {a_1}{{{\lambda}_1}^m}{x_1} + {a_2}{{{\lambda}_2}^m}{x_2} + .... + {a_n}{{{\lambda}_n}^m}{x_n}$\\
Let us assume, ${{\lambda}_1}$ is the largest Eigen value. Hence on dividing by ${{{\lambda}_1}^m}$ the expression,we get\\
$ = {{{\lambda}_1}^m}({a_1}{x_1} + \frac{{{\lambda}_2}^m}{{{\lambda}_1}^m}{a_2}{x_2} + .... + \frac{{{\lambda}_n}^m}{{{\lambda}_1}^m}{a_n}{x_n})$ -Eqn.(2)\\
Since ${{\lambda}_1}$ is the largest Eigen value, $\left|\frac{{\lambda}_j}{{\lambda}_1}\right| < 1$ for all $2 \leq j \leq n$\\
As $m \rightarrow \infty$,  $\left|\frac{{{\lambda}_j}^m}{{{\lambda}_1}^m}\right| \rightarrow 0$ -Eqn.(3) for all $2 \leq j \leq n$\\
\\
Using this, we can rewrite the Eqn.(2) as\\
${A^m}{z^{(0)}} = {{{\lambda}_1}^m}({a_1}{x_1} + \frac{{{\lambda}_2}^m}{{{\lambda}_1}^m}{a_2}{x_2} + .... + \frac{{{\lambda}_n}^m}{{{\lambda}_1}^m}{a_n}{x_n})$\\
$ = {{{\lambda}_1}^m}({a_1}{x_1} + \sum_{j=2}^{n}\frac{{{\lambda}_j}^m}{{{\lambda}_1}^m}{a_j}{x_j})$\\
$ = {{{\lambda}_1}^m}({a_1}{x_1} + \sum_{j=2}^{n}0.{a_j}{x_j})$\\
$ =  {{{\lambda}_1}^m}{a_1}{x_1}$\\
\\
Here $a$ is an arbitrary constant which doesn't affect the vector $x_1$.
Hence we can say, ${A^m}{z^{(0)}} \rightarrow {{{\lambda}_1}^m}{x_1}$\\
\\
{\bf b)} ${{{\lambda}_1}^{(m)}} =  \frac{{w_k}^{(m)}}{{z_k}^{(m)}}$ -Eqn.(1) and ${{\lambda}_1}$ is the largest Eigen value.
i.e $\left|{{\lambda}_1} \right| > \left|{{\lambda}_2} \right| > ..... > \left|{{\lambda}_n} \right|$.\\
$z^{(0)} = {c_1}{v^{(1)}} + {c_2}{v^{(2)}} + ...+ {c_n}{v^{(n)}}$ where $c_1, c_2, ... c_n$ are constants and $v^{(1)}, v^{(2)}, .. v^{(n)}$ are the Eigen vectors\\
$z^{(m - 1)} = A^{m - 1}z^{(0)}$\\
$ = \sum_{i=1}^{n}{c_i}{{{\lambda}_i}^{(m-1)}}{v^{(i)}}$\\
$ = {{{\lambda}_1}^{(m-1)}}({c_1}{v^{(1)}} + \sum_{i=2}^{n}\frac{{{\lambda}_i}^{(m-1)}}{{{\lambda}_1}^{(m-1)}}{c_i}{v^{(i)}})$ -Eqn.(3)\\
\\
$w^m = Az^{(m - 1)}$\\
$ = A\sum_{i=1}^{n}{c_i}{{{\lambda}_i}^{(m-1)}}{v^{(i)}}$\\
$ = \sum_{i=1}^{n}{c_i}{{{\lambda}_i}^{(m-1)}}A{v^{(i)}}$\\
$ = \sum_{i=1}^{n}{c_i}{{{\lambda}_i}^{(m-1)}}{{\lambda}_i}{v^{(i)}}$  As we know from the Eigen value property that $A{v^{(i)}} = {{\lambda}_i}{v^{(i)}}$\\
$ = \sum_{i=1}^{n}{c_i}{{{\lambda}_i}^{(m)}}{v^{(i)}}$\\
$ = {c_1}{{{\lambda}_1}^{(m)}}{v^{(1)}} + {c_2}{{{\lambda}_2}^{(m)}}{v^{(2)}} + ....  + {c_n}{{{\lambda}_n}^{(m)}}{v^{(n)}}$\\
$ = {{{\lambda}_1}^{(m)}}({c_1}{v^{(1)}} + \frac{{{\lambda}_2}^{(m)}}{{{\lambda}_1}^{(m)}}{c_2}{v^{(2)}} + ....  +  \frac{{{\lambda}_n}^{(m)}}{{{\lambda}_1}^{(m)}}{c_n}{v^{(n)}})$ -Eqn.(3)\\
\\
From the Eqns (1), (2),(3), we can write,\\
${{{\lambda}_1}^{(m)}} = \frac{{{{\lambda}_1}^{(m)}}({c_1}{v^{(1)}} + \frac{{{\lambda}_2}^{(m)}}{{{\lambda}_1}^{(m)}}{c_2}{v^{(2)}} + ....  +  \frac{{{\lambda}_n}^{(m)}}{{{\lambda}_1}^{(m)}}{c_n}{v^{(n)}})}{{{{\lambda}_1}^{(m-1)}}({c_1}{v^{(1)}} + \frac{{{\lambda}_2}^{(m-1)}}{{{\lambda}_1}^{(m-1)}}{c_2}{v^{(2)}} + ....  +  \frac{{{\lambda}_n}^{(m-1)}}{{{\lambda}_1}^{(m-1)}}{c_n}{v^{(n)}})}$ -Eqn.(4)\\
Since $\left|{{\lambda}_1} \right| > \left|{{\lambda}_2} \right| > ..... > \left|{{\lambda}_n} \right|$, as $m \rightarrow \infty$, we have, $\left|\frac{{{\lambda}_j}^m}{{{\lambda}_1}^m}\right| \rightarrow 0$ and  $\left|\frac{{{\lambda}_j}^{(m-1)}}{{{\lambda}_1}^{(m-1)}}\right| \rightarrow 0$ for all $2 \leq j \leq n$\\
\\
Hence Eqn.(4) is reduced to ${{{\lambda}_1}^{(m)}} = {{\lambda}_1}\frac{({c_1}{v^{(1)}} + \frac{{{\lambda}_2}^{(m)}}{{{\lambda}_1}^{(m)}}{c_2}{v^{(2)}} + ....  +  \frac{{{\lambda}_n}^{(m)}}{{{\lambda}_1}^{(m)}}{c_n}{v^{(n)}})}{({c_1}{v^{(1)}} + \frac{{{\lambda}_2}^{(m-1)}}{{{\lambda}_1}^{(m-1)}}{c_2}{v^{(2)}} + ....  +  \frac{{{\lambda}_n}^{(m-1)}}{{{\lambda}_1}^{(m-1)}}{c_n}{v^{(n)}})}$\\
$ \approx {{\lambda}_1}\frac{({c_1}{v^{(1)}} + 0.{c_2}{v^{(2)}} + ....  +  0.{c_n}{v^{(n)}})}{({c_1}{v^{(1)}} + 0.{c_2}{v^{(2)}} + ....  +  0.{c_n}{v^{(n)}})}$\\
$ \approx {{\lambda}_1}\frac{({c_1}{v^{(1)}})}{({c_1}{v^{(1)}})}$\\
$ \approx {{\lambda}_1}$\\
\\
Hence ${{{\lambda}_1}^{(m)}} \rightarrow {{\lambda}_1}$ as $m \rightarrow \infty$.\\
\\
{\bf c)} Refer to the program {\it q1test.m}.\\
\\
{\bf d)} Using the program we found out the the largest Eigen value to be: 9.6235. Attaching below the output snippet from matlab\\
Enter the matrix: [1 2 3; 2 3 4; 3 4 5]\\
Enter vector z: [1 1 1]'\\
Enter value of n: 50\\
The largest eigen value for this matrix is:\\ 
    9.6235\\
\\
The corresponding eigen vector for this matrix is:\\ 
    0.5247\\
    0.7623\\
    1.0000\\
\\
Total number of iterations:\\ 
     8\\
\\
{\bf e)}
A = $\begin{bmatrix}
1   &     2&       3\\
   2  &  3    &    4\\
    3  &  4 &   5 \\
\end{bmatrix}$ \\
\\
For Eigen value ad Eigen vector property, we know that $Ax = {\lambda}x$ and $Av = {\lambda}v$ where ${\lambda}$ is the Eigen value and $v$ is the Eigen vector.\\
The determinant of $(A - {\lambda}I)$ must be equal to 0 for the matrix to have non-zero Eigen vectors.\\
$\begin{bmatrix}
1   &     2&       3\\
   2  &  3    &    4\\
    3  &  4 &   5 \\
\end{bmatrix}$  - $\begin{bmatrix}
{\lambda}   &     0&       0\\
   0  &  {\lambda}    &    0\\
    0  &  0 &   {\lambda}\\
\end{bmatrix} = 0$\\
$\implies$ det$\begin{bmatrix}
1 - {\lambda}   &     2&       3\\
   2  &  3 - {\lambda}    &    4\\
    3  &  4 &   5 - {\lambda} \\
\end{bmatrix}  = 0$\\
$\implies$ $(1 - {\lambda})[(3 - {\lambda})(5 - {\lambda}) - 16] + 2[12 - 2(5 - {\lambda})] + 3[8 - 3(3 - {\lambda})] = 0$\\
$\implies (1 - {\lambda})[15 + {\lambda}^2 - 8{\lambda} - 16] + 2[12 - 10 + 2{\lambda}] + 3[8 - 9 + 3{\lambda}] = 0$\\
$\implies (1 - {\lambda})[{\lambda}^2 - 8{\lambda} - 1] + 4[1 + {\lambda}] + 3[3{\lambda} - 1] = 0$\\
$\implies {\lambda}^2 - 8{\lambda} - 1 - {\lambda}^3 + 8{\lambda}^2 + {\lambda} + 4[1 + {\lambda}] + 3[3{\lambda} - 1]  = 0$\\
$\implies - {\lambda}^3 + 9{\lambda}^2 + 6{\lambda} = 0$\\
$\implies {\lambda}^3 - 9{\lambda}^2 - 6{\lambda} = 0$\\
$\implies {\lambda}[{\lambda}^2 - 9{\lambda} - 6] = 0$\\
\\
Solving this, we get the roots: ${\lambda} = 0, -0.6235, 9.6235$.
There are total 3 Eigen values ${\lambda}_1 = 9.6235, {\lambda}_2 = -0.6235, {\lambda}_3 = 0$.\\
\\
We need to find out the normalized Eigen vectors.
We have given the procedure below:\\
For ${\lambda}_1 = 9.6235$,
$(A - {{\lambda}_1}I)v^{(1)} = 0$\\
$\implies \begin{bmatrix}
1 - 9.6235   &     2&       3\\
   2  &  3 - 9.6235    &    4\\
    3  &  4 &   5 - 9.6235 \\
\end{bmatrix}\begin{bmatrix}
v_1 \\
   v_2\\
    v_3 \\
\end{bmatrix} = 0$\\
Let's take $v_1$ = 1, 
We get three equations from the matrix above. The equations are:\\
$2v_2 + 3v_3 = 8.6235$ -Eqn.(1)\\
$-6.6235v_2 + 4v_3 = -2$ -Eqn.(2)\\
$4v_2 - 4.6235v_3 = -3$ -Eqn.(3)\\
\\
Solving this we get, $v_2 = 1.4529$, $v_3 = 1.9059$ \\
Hence, 
$v^{(1)} =  \begin{bmatrix}
\frac{v_1}{\sqrt{{v_1}^2 + {v_2}^2} + {v_3}^2} \\
\\
   \frac{v2}{\sqrt{{v_1}^2 + {v_2}^2} + {v_3}^2}\\
\\
    \frac{v_3}{\sqrt{{v_1}^2 + {v_2}^2} + {v_3}^2} \\
\end{bmatrix} = \begin{bmatrix}
0.3851 \\
  0.5595\\
    0.7339 \\
\end{bmatrix}$\\
\\
\\
For ${\lambda}_2 = -0.6235$, let's solve for corrsponding eigen vector $v^{(2)}$ using $(A - {{\lambda}_2}I)v^{(2)} = 0$\\We have the matrix,\\
$\begin{bmatrix}
1 + 0.6235   &     2&       3\\
   2  &  3 + 0.6235    &    4\\
    3  &  4 &   5 + 0.6235 \\
\end{bmatrix}\begin{bmatrix}
v_1 \\
   v_2\\
    v_3 \\
\end{bmatrix} = 0$\\
we have the following equations after setting $v_1$ to 1,\\
\\
$2v_2 + 3v_3 = -1.6235$ -Eqn.(4)\\
$3.6235v_2 + 4v_3 = -2$ -Eqn.(5)\\
$4v_2 + 5.6235v_3 = -3$ -Eqn.(6)\\
\\
Solving which we get $v_2 = 0.1723$ and $v_3 = -.6560$.Using these values
We have the normalized Eigen vector\\ $v^{(2)} =  \begin{bmatrix}
\frac{v_1}{\sqrt{{v_1}^2 + {v_2}^2} + {v_3}^2} \\
\\
   \frac{v2}{\sqrt{{v_1}^2 + {v_2}^2} + {v_3}^2}\\
\\
    \frac{v_3}{\sqrt{{v_1}^2 + {v_2}^2} + {v_3}^2} \\
\end{bmatrix} = \begin{bmatrix}
0.8277 \\
  0.1424\\
    -0.5428 \\
\end{bmatrix}$\\
\\
\\
For ${\lambda}_3 = 0$, let's solve for corrsponding eigen vector $v^{(3)}$ using $(A - {{\lambda}_3}I)v^{(3)} = 0$\\We have the matrix,\\
$\begin{bmatrix}
1   &     2&       3\\
   2  &  3    &    4\\
    3  &  4 &   5 \\
\end{bmatrix}\begin{bmatrix}
v_1 \\
   v_2\\
    v_3 \\
\end{bmatrix} = 0$\\
we have the following equations after setting $v_1$ to 1,\\
$2v_2 + 3v_3 = -1$\\
$3v_2 + 4v_3 = -2$\\
$4v_2 + 5v_3 = -3$\\
\\
Solving which we get $v_2 = -2$ and $v_3 = 1$.Using these values we figure out that for the Eigen value ${\lambda}_3 = 0$, the corresponding normlized Eigen vector $v^{(3)} = =  \begin{bmatrix}
\frac{v_1}{\sqrt{{v_1}^2 + {v_2}^2} + {v_3}^2} \\
\\
   \frac{v2}{\sqrt{{v_1}^2 + {v_2}^2} + {v_3}^2}\\
\\
    \frac{v_3}{\sqrt{{v_1}^2 + {v_2}^2} + {v_3}^2} \\
\end{bmatrix} = \begin{bmatrix}
0.4082 \\
  -0.8165\\
    0.4082 \\
\end{bmatrix}$\\
\end{document}